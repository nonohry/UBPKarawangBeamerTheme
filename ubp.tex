%% Do not edit unless you really know what you are doing.
\documentclass[english]{beamer}
\usepackage[T1]{fontenc}
\usepackage[latin9]{inputenc}
\setcounter{secnumdepth}{3}
\setcounter{tocdepth}{3}
\definecolor{note_fontcolor}{rgb}{0.800781, 0.800781, 0.800781}

\makeatletter

%%%%%%%%%%%%%%%%%%%%%%%%%%%%%%
\providecommand{\LyX}{L\kern-.1667em\lower.25em\hbox{Y}\kern-.125emX\@}
%% The greyedout annotation environment
\newenvironment{lyxgreyedout}
  {\textcolor{note_fontcolor}\bgroup\ignorespaces}
  {\ignorespacesafterend\egroup}

%%%%%%%%%%%%%%%%%%%%%%%%%%%%%% Textclass specific LaTeX commands.
 % this default might be overridden by plain title style
 \newcommand\makebeamertitle{\frame{\maketitle}}%
 % (ERT) argument for the TOC
 \AtBeginDocument{%
   \let\origtableofcontents=\tableofcontents
   \def\tableofcontents{\@ifnextchar[{\origtableofcontents}{\gobbletableofcontents}}
   \def\gobbletableofcontents#1{\origtableofcontents}
 }
 \newenvironment{lyxcode}
   {\par\begin{list}{}{
     \setlength{\rightmargin}{\leftmargin}
     \setlength{\listparindent}{0pt}% needed for AMS classes
     \raggedright
     \setlength{\itemsep}{0pt}
     \setlength{\parsep}{0pt}
     \normalfont\ttfamily}%
    \def\{{\char`\{}
    \def\}{\char`\}}
    \def\textasciitilde{\char`\~}
    \item[]}
   {\end{list}}

%%%%%%%%%%%%%%%%%%%%%%%%%%%%%% User specified LaTeX commands.
\mode<presentation>
{
  %FrankfurtLuebeckMadridMalmoePittsburghWarsaw
 \usetheme{Frankfurt}
  %\usecolortheme{default}
  %\usepackage{beamerthemesplit}% or try albatross, beaver, crane, ...
  %\usefonttheme{default}  % or try serif, structurebold, ...
  %\setbeamertemplate{navigation symbols}{}
  \setbeamertemplate{caption}[numbered]
} 

%\beamertemplatenavigationsymbolsempty 
%\usetheme[secheader,compress]{Madrid} %Primary theme
\setbeamertemplate{headline}{}  % uncomment this line if do not want headline shown in every frame.

\logo{\includegraphics[height=1.0cm]{img/ubp.png}}
% Change base colour beamer@blendedblue (originally RGB: 0.2,0.2,0.7)

% Menambahkan Background
\usepackage{tikz}
\usebackgroundtemplate{%
\tikz\node[opacity=0.075]
{\includegraphics[height=\paperheight,width=\paperwidth]{img/ubpbg.png}};}
%{\includegraphics[height=9cm,width=9cm]{logo.png}};}

\colorlet{beamer@blendedblue}{cyan!95!black}

\usepackage{listings}
\lstloadlanguages{Python}

\usepackage{alltt}
\usepackage{amssymb}
\usepackage{amsmath}
%\usepackage{beamerprosper}
\usepackage[english]{babel}
\usepackage{booktabs}
\usepackage{calc}
\usepackage{colortbl}
\usepackage{helvet}
\usepackage{mathptmx}
\usepackage{multirow}
\usepackage{pgf}
\usepackage{smartdiagram}
\usepackage{tabularx}
\usepackage{tikz}
\usepackage{ulem}
\usepackage{xmpmulti}

\DeclareGraphicsRule{*}{mps}{*}{}
\DeclareMathOperator*{\argmax}{\arg\max}

\usetikzlibrary{%
  arrows,%
  automata,%
  calc,%
  trees,%
  positioning,%
  chains,%
  shapes,%
  shapes.arrows,%
  shapes.geometric,%
  shapes.misc,% wg. rounded rectangle
  shapes.symbols,%
  decorations.pathreplacing,%
  decorations.pathmorphing,% /pgf/decoration/random steps | erste Graphik
  matrix,%
  scopes,%
  shadows%
 }

%\includeonlyframes{cur}

\pgfdeclarelayer{background}
\pgfdeclarelayer{foreground}
\pgfsetlayers{background,main,foreground}

%PDF Information
\title{Nono Heryana}
\subject{nono@unsika.ac.id}
\author{Nono Heryana}
\keywords{nono, ganteng}

%header
\useoutertheme[subsection=false]{miniframes} % Alternatively: miniframes, infolines, split
\useinnertheme{circles}

\makeatother

\usepackage{babel}
\begin{document}

\title{Membuat Presentasi dengan \LyX{}}


\subtitle{\LyX{} emang keren!}


\author[Nono Heryana]{Nono Heryana\\
\alert{\texttt{nonoheryana@gmail.com}}}


\institute[UBP Karawang]{Fakultas Teknologi dan Ilmu Komputer\\
Universitas Buana Perjuangan Karawang}


\date{Karawang, 2016}
\makebeamertitle
\begin{frame}{Outline}


\tableofcontents{}
\end{frame}



\section{Intro}


\subsection{Latar Belakang}
\begin{frame}{Latar Belakang}


\begin{lyxgreyedout}
latar belakang%
\end{lyxgreyedout}

\end{frame}

\subsection{Rumusan Masalah}
\begin{frame}{Rumusan Masalah}


\begin{lyxgreyedout}
Rumusan Masalah%
\end{lyxgreyedout}

\end{frame}

\section{Metode}


\subsection{Tinjauan Studi}
\begin{frame}{Tinjauan Studi}


\begin{lyxgreyedout}
Tinjauan Studi%
\end{lyxgreyedout}



\end{frame}

\subsection{Metode}
\begin{frame}{Metode}


\begin{lyxgreyedout}
Metode%
\end{lyxgreyedout}

\end{frame}

\section{Hasil}


\subsection{Hasil}
\begin{frame}{Hasil}


\begin{lyxgreyedout}
Hasil%
\end{lyxgreyedout}

\end{frame}

\subsection{Pengujian}
\begin{frame}{Pengujian}


\begin{lyxgreyedout}
Pengujian%
\end{lyxgreyedout}

\end{frame}

\section{Kesimpulan}
\begin{frame}{Kesimpulan}


\begin{lyxgreyedout}
Kesimpulan%
\end{lyxgreyedout}

\end{frame}

\begin{frame}[plain]{Thanks}

\begin{lyxcode}
feedback~=~\textcolor{blue}{raw\_input}~(~\textcolor{red}{\textquoteright Questions~?\textquoteright{}}~)~

\textcolor{blue}{if}~\textcolor{red}{\textquoteright ?\textquoteright{}}~\textcolor{blue}{in}~feedback~:~
\begin{lyxcode}
\textcolor{blue}{if}~have\_answer~():~
\begin{lyxcode}
give\_answer~()~
\end{lyxcode}
\textcolor{blue}{else}:~
\begin{lyxcode}
pretend\_the\_question\_is\_ill\_posed~()~
\end{lyxcode}
\end{lyxcode}
\textcolor{blue}{else}:~
\begin{lyxcode}
\textcolor{blue}{print}~\textcolor{red}{\textquoteright Thanks~for~your~attention.\textquoteright{}}\end{lyxcode}
\end{lyxcode}
\end{frame}

\end{document}